\documentclass[11pt, a4paper, leqno]{article}
\usepackage{a4wide}
\usepackage[T1]{fontenc}
\usepackage[utf8]{inputenc}
\usepackage{float, afterpage, rotating, graphicx}
\usepackage{epstopdf}
\usepackage{longtable, booktabs, tabularx}
\usepackage{fancyvrb, moreverb, relsize}
\usepackage{eurosym, calc}
% \usepackage{chngcntr}
\usepackage{amsmath, amssymb, amsfonts, amsthm, bm}
\usepackage{caption}
\usepackage{mdwlist}
\usepackage{xfrac}
\usepackage{setspace}
\usepackage[dvipsnames]{xcolor}
\usepackage{subcaption}
\usepackage{minibox}
% \usepackage{pdf14} % Enable for Manuscriptcentral -- can't handle pdf 1.5
% \usepackage{endfloat} % Enable to move tables / figures to the end. Useful for some
% submissions.
\usepackage[colorlinks=true, allcolors=blue]{hyperref}

\usepackage[
    natbib=true,
    bibencoding=inputenc,
    bibstyle=authoryear-ibid,
    citestyle=authoryear-comp,
    maxcitenames=3,
    maxbibnames=10,
    useprefix=false,
    sortcites=true,
    backend=biber
]{biblatex}
\AtBeginDocument{\toggletrue{blx@useprefix}}
\AtBeginBibliography{\togglefalse{blx@useprefix}}
\setlength{\bibitemsep}{1.5ex}
\addbibresource{../../paper/refs.bib} % fix this

\widowpenalty=10000
\clubpenalty=10000

\setlength{\parskip}{1ex}
\setlength{\parindent}{0ex}
\setstretch{1.5}

\title{Time Series Forecasting\thanks{Julio Alonzo, University of Bonn. Email: \href{mailto:j.c.alonzo.muller@gmail.com}{\nolinkurl{j.c.alonzo.muller@gmail.com}}. Template from \cite{GaudeckerEconProjectTemplates}}}
\author{Julio Alonzo}

\date{
    {\bf Preliminary -- please do not quote}
    \\[1ex]
    \today
}

\begin{document}
\maketitle

\begin{abstract}
This is a project where a couple of time series forecasting methods are applied to unemployment rate data.
\end{abstract}

\section{Introduction}

Time series are highly relevant in economics. It is of interest to have methods that allow us to forecast their development.
I assume that if one were to formulate a research question for this project it would be along the lines of: how does the performance of time series models vary across economic indicators such as the unemployment rate?
Of course I don't pretend to provide a particularly rigorous answer to the aforementioned question, which could very well be the task of a master thesis. Here I pretend no more than to document my learning process of the basics of time series analysis\footnote{Task for which \cite{huang2022applied} came in handy.} and how to use python as a tool in this context.

The data was taken from \url{https://fred.stlouisfed.org/series/UNRATE}. The data was already clean so no complex data cleaning function waas required.

\section{Graphs and Tables}

\begin{figure}[H]

    \centering
    \includegraphics[width=0.8\textwidth]{../python/figures/hw_forecasts.png}

    \caption{Forecasted values using the \emph{Holt-Winters} method plotted against the actual data.}
    \label{fig:hw}

\end{figure}

\begin{figure}[H]

    \centering
    \includegraphics[width=0.8\textwidth]{../python/figures/arima_forecasts.png}

    \caption{Forecasted values using the \emph{ARIMA} method plotted against the actual data.}
    \label{fig:arima}

\end{figure}

\begin{figure}[H]

    \centering
    \includegraphics[width=0.8\textwidth]{../python/figures/metrics.png}

    \caption{Different measures of the accuracy of both methods employed for forecasting.}
    \label{fig:measures}

\end{figure}



\input{../python/tables/metrics.tex}

\setstretch{1}
\printbibliography
\setstretch{1.5}


\end{document}
